% Documento simplificado para una tesis tipo MIT
% Versión más simple con explicaciones contextuales

\documentclass[11pt]{report} % Clase de documento adecuada para tesis

% Paquetes esenciales
\usepackage[letterpaper, margin=1in]{geometry} % Configuración de márgenes
\usepackage{graphicx} % Paquete para incluir gráficos
\usepackage{amsmath} % Paquete para ecuaciones matemáticas avanzadas
\usepackage{amssymb} % Símbolos matemáticos adicionales
\usepackage{setspace} % Espaciado entre líneas
\usepackage{titlesec} % Personalizar los títulos de secciones

% Información de la portada
\title{Sistema de carga modular y aterrizaje de precisión para enjambre de drones} % Título de la tesis
\author{José Alberto Castro Villasana \and José Eduardo Castro Villasana \and Ana Bárbara Quintero García} % Nombres de los autores
\date{Diciembre 2024} % Fecha de presentación

\begin{document}

% Portada
\begin{titlepage}
    \centering
    \vspace*{1cm}
    
    {\Huge\textbf{Sistema de carga modular y aterrizaje de precisión para enjambre de drones}} % Título en grande
    
    \vspace{2.5cm} % Espacio adicional para centrar mejor
    {\Large\textbf{José Alberto Castro Villasana, José Eduardo Castro Villasana, Ana Bárbara Quintero García}} % Nombres de los autores
    
    \vfill
    
    {\large\textbf{Una Tesis presentada a la Facultad de Ingeniería}} % Información sobre la presentación
    {\large\textbf{en Conformidad con los Requisitos para el Grado de}} % Información sobre la presentación
    {\large\textbf{Ingeniero en Tecnologías Electrónicas y Robótica, Ingeniero en Mecatrónica}} % Información sobre la presentación
    
    \vspace{2cm}
    
    \includegraphics[width=0.4\textwidth]{pictures/logo_udem.png} % Logo de la institución
    
    \vspace{2cm}
    
    {\large\textbf{Universidad de Monterrey} \newline Departamento de Ingeniería y Tecnologías \newline San Pedro Garza García, México} % Institución y departamento
    
    \vspace{1.5cm}
    {\large\textbf{Diciembre 2024}} % Fecha de presentación
    
    \vfill
    {\large\textbf{Asesor:} \textbf{Fermín Castro}} % Información del asesor
    
\end{titlepage}

% Índice
\tableofcontents
\newpage

% Espaciado entre líneas
\onehalfspacing% Espaciado de 1.5 líneas para mejor legibilidad

% Llamar a capítulos almacenados en archivos externos
\chapter{Introducción} % Capítulo de introducción
% Introducción para la tesis
% Documento independiente para la introducción

\section{Contexto General y Motivación}

\subsection{Contexto del Problema}

En los últimos años, los drones han evolucionado para convertirse en herramientas esenciales en una amplia gama de industrias. Estos vehículos aéreos no tripulados varían en tamaño y capacidad, pero comparten la característica de acceder a áreas difíciles o peligrosas, lo que ha impulsado su uso en sectores como la inspección de infraestructuras, la agricultura de precisión y las operaciones de vigilancia y seguridad. A nivel comercial, los drones están diseñados para realizar tareas repetitivas de manera autónoma, lo que ha revolucionado la eficiencia en dichas áreas.

En el ámbito de la inspección de infraestructuras, los drones son utilizados para monitorear y evaluar el estado de torres eléctricas, aerogeneradores, líneas de transmisión y otras construcciones críticas, permitiendo detectar fallos estructurales sin la necesidad de exponer a los trabajadores a riesgos innecesarios \cite{cite1}. En agricultura, los drones facilitan la gestión de cultivos, monitoreo de riego y la topografía, lo que permite a los agricultores optimizar la productividad y reducir costos operativos \cite{cite2}. En vigilancia y seguridad, se emplean para monitorear incendios forestales, realizar labores de seguridad fronteriza y evaluar daños tras desastres naturales \cite{cite3}.

A pesar de sus múltiples ventajas, uno de los mayores desafíos en la operación continua de drones es la duración limitada de la batería. En aplicaciones profesionales, los drones suelen tener una autonomía de vuelo entre 20 y 30 minutos, lo cual resulta insuficiente para operaciones prolongadas como la inspección de grandes áreas o la vigilancia a largo plazo \cite{cite4}. Además, las condiciones climáticas adversas pueden afectar significativamente el rendimiento y estabilidad de los drones en vuelo, lo que limita aún más su capacidad operativa \cite{cite5}. Estos factores hacen necesario interrumpir las operaciones para recargar las baterías, lo cual afecta tanto la eficiencia como los costos operativos.

Para abordar esta limitación, se han diseñado estaciones de carga automáticas que permiten recargar las baterías sin intervención humana. Sin embargo, las estaciones actuales presentan varios inconvenientes: la mayoría no están diseñadas para gestionar múltiples drones de manera simultánea, lo que limita su eficiencia en operaciones a gran escala, como las que utilizan enjambres de drones \cite{cite6}. Además, el aterrizaje preciso sobre estas estaciones sigue siendo un reto importante, lo que puede afectar el éxito del proceso de carga \cite{cite7}.

\subsection{Importancia del Problema}

Con el creciente uso de drones en sectores como la agricultura, la logística y la vigilancia, la capacidad de mantener operaciones prolongadas y autónomas se ha vuelto crucial. Estos sectores demandan drones capaces de completar misiones largas sin interrupciones frecuentes para la recarga de baterías \cite{cite8}.

Aquí es donde entra en juego la necesidad de un sistema de carga modular. En este contexto, la modularidad se refiere a la posibilidad de montar varios sistemas de carga en un mismo espacio físico, lo que permite a múltiples drones recargarse simultáneamente. Esto optimiza el uso de los recursos y maximiza la eficiencia operativa de los enjambres de drones, al reducir los tiempos de espera entre recargas \cite{cite9}. Esta solución es especialmente valiosa en aplicaciones que requieren la coordinación de varios drones en tiempo real, donde cualquier retraso en la recarga podría impactar negativamente en la misión.

Además, un sistema de aterrizaje preciso es clave para garantizar que los drones se posen correctamente sobre las estaciones de carga, lo que permite automatizar completamente el proceso y reducir la necesidad de intervención humana. Este tipo de solución tiene el potencial de mejorar la eficiencia y autonomía de las operaciones con drones, abriendo nuevas posibilidades para su adopción en áreas donde las limitaciones actuales de autonomía y recarga son un obstáculo \cite{cite10}.

\section{Planteamiento del Problema}

\subsection{Definición Clara del Problema}

Uno de los mayores desafíos para la operación continua y autónoma de enjambres de drones es la falta de estaciones de carga modulares y autónomas que permitan la recarga simultánea de múltiples drones. Actualmente, las soluciones de carga disponibles en el mercado presentan limitaciones significativas, como la imposibilidad de gestionar de manera eficiente la carga de varios drones al mismo tiempo, y que aseguren un aterrizaje preciso en un mismo espacio sin intervención humana. Estas limitaciones no solo disminuyen la autonomía de los drones, sino que también afectan la eficiencia operativa, ya que requieren intervenciones frecuentes para maniobrar los drones de manera manual y requiere mayor espacio en casi de utilizar las soluciones que se encuentran actualmente en el mercado.
\subsection{Preguntas de Investigación}

- ¿Cómo se puede desarrollar una estación de carga modular que permita la carga simultánea de múltiples drones?
- ¿Qué tecnologías de localización y visión pueden ser implementadas para asegurar un aterrizaje preciso?
- ¿Cuales serían los recursos necesarios para implementar una estación de carga y aterrizaje de precisión en un enjambre de drones?
- ¿Qué tipo de carga es la más adecuada para la carga de un dron en cuestión de eficiencia y rendimiento?

\section{Objetivos}

\subsection{Objetivo General}

Diseñar y manufacturar una base de carga para cuadricópteros de arquitectura abierta, desarrollar un sistema de comunicación entre el drone y la estación de carga, e instrumentar el drone con de sensores de localización y un sistema de visión, para asegurar una integración eficaz y segura entre el cuadricóptero y su sistema de carga.

\subsection{Objetivos Específicos}

\begin{itemize}
    \item Realizar un diseño CAD para la base de carga, asegurando que sea compatible con cuadricópteros de arquitectura abierta.
    \item Manufacturar una base de carga para el enjambre de drones.
    \item Adaptar el diseño actual del dron de arquitectura abierta a la base de carga.
    \item Instrumentar un cuadricóptero con oboard computer, flight controler, cámara de visión and sensores de localización.
    \item Programar un sistema de comunicación entre la base de carga y el cuadricóptero para la trasnferencia de datos relevantes para su interacción.
    \item Programar un sistema que permita controlar el drone desde una onboard computer.
    \item Programar un sistema de visión por computadora que detecte en tiempo real la pose de un Aruco Marker embebido en la base de carga.
    %\item Complementar sistemas de control de motores con el sistema de vision por computadora para lograr un aterrizaje preciso.
\end{itemize}

\section{Justificación}

\subsection{Relevancia del Proyecto}

La operación continua y autónoma de enjambres de drones enfrenta varios obstáculos, entre ellos, la falta de soluciones eficientes de carga que permitan la recarga simultánea de múltiples drones de manera autónoma y sin intervención humana. Aunque este proyecto no busca solucionar completamente este desafío, sí establece las bases fundamentales para desarrollar una solución viable. Al diseñar una estación de carga modular y autónoma con un sistema de aterrizaje preciso, este trabajo proporciona un punto de partida sobre el cual se pueden realizar futuras investigaciones y mejoras.

El proyecto es relevante para sectores como la logística, la vigilancia y la agricultura de precisión, donde la operación autónoma de drones puede generar grandes beneficios. Al presentar una solución modular que permite recargar varios drones en un mismo espacio físico, se sientan las bases para una futura implementación a mayor escala, lo que potencialmente podría reducir los costos operativos y mejorar la eficiencia de las operaciones \cite{cite1}. Aunque no se pretende resolver el problema de manera completa en este trabajo, los avances presentados contribuirán al desarrollo de soluciones reales en el futuro.

\subsection{Impacto Potencial}

A pesar de que el sistema desarrollado en este proyecto no es una solución final, su implementación tiene el potencial de influir en futuras investigaciones y desarrollos en el campo de la autonomía de drones. Las tecnologías de aterrizaje preciso y carga simultánea que se exploran aquí podrían ser fundamentales para el diseño de soluciones más avanzadas y escalables, capaces de abordar las necesidades operativas de sectores que dependen cada vez más de los drones.

El impacto potencial de este proyecto se reflejará en la capacidad de inspirar investigaciones que mejoren la eficiencia de los enjambres de drones en aplicaciones como la entrega de mercancías, la vigilancia de grandes áreas y la optimización de recursos en la agricultura. Si bien las soluciones actuales enfrentan limitaciones, este trabajo ofrece una plataforma sólida sobre la cual se podrían implementar mejoras significativas en futuras etapas \cite{cite2}.


Memoria actualizada
Gracias por la aclaración. Aquí tienes una versión ajustada de las secciones, teniendo en cuenta que el proyecto establece las bases para una solución real:

\section{Justificación}

\subsection{Relevancia del Proyecto}

La operación continua y autónoma de enjambres de drones enfrenta varios obstáculos, entre ellos, la falta de soluciones eficientes de carga que permitan la recarga simultánea de múltiples drones de manera autónoma y sin intervención humana. Aunque este proyecto no busca solucionar completamente este desafío, sí establece las bases fundamentales para desarrollar una solución viable. Al diseñar una estación de carga modular y autónoma con un sistema de aterrizaje preciso, este trabajo proporciona un punto de partida sobre el cual se pueden realizar futuras investigaciones y mejoras.

El proyecto es relevante para sectores como la logística, la vigilancia y la agricultura de precisión, donde la operación autónoma de drones puede generar grandes beneficios. Al presentar una solución modular que permite recargar varios drones en un mismo espacio físico, se sientan las bases para una futura implementación a mayor escala, lo que potencialmente podría reducir los costos operativos y mejorar la eficiencia de las operaciones \cite{cite1}. Aunque no se pretende resolver el problema de manera completa en este trabajo, los avances presentados contribuirán al desarrollo de soluciones reales en el futuro.

\subsection{Impacto Potencial}

A pesar de que el sistema desarrollado en este proyecto no es una solución final, su implementación tiene el potencial de influir en futuras investigaciones y desarrollos en el campo de la autonomía de drones. Las tecnologías de aterrizaje preciso y carga simultánea que se exploran aquí podrían ser fundamentales para el diseño de soluciones más avanzadas y escalables, capaces de abordar las necesidades operativas de sectores que dependen cada vez más de los drones.

El impacto potencial de este proyecto se reflejará en la capacidad de inspirar investigaciones que mejoren la eficiencia de los enjambres de drones en aplicaciones como la entrega de mercancías, la vigilancia de grandes áreas y la optimización de recursos en la agricultura. Si bien las soluciones actuales enfrentan limitaciones, este trabajo ofrece una plataforma sólida sobre la cual se podrían implementar mejoras significativas en futuras etapas \cite{cite2}.

\section{Alcance y Limitaciones}

\subsection{Alcance del Trabajo}

Este proyecto establece las bases para una solución modular y autónoma de recarga de drones. Se enfoca en el diseño, manufactura e instrumentación de una estación de carga que permita gestionar múltiples drones simultáneamente, utilizando tecnologías de visión por computadora y sensores de localización. El trabajo se llevará a cabo dentro de un entorno controlado y se validará en condiciones simuladas para asegurar que la solución es viable desde un punto de vista técnico.

Aunque el proyecto no resuelve el problema de la carga autónoma a gran escala, proporciona un primer paso hacia una solución más avanzada. Se presentará el diseño de un sistema de carga modular que permita recargar varios drones en el mismo espacio, sentando así las bases para desarrollos posteriores, y físicamente se realizará un prototipo funcional de uno de los modulos de carga para un drone. Las pruebas se realizarán en un cuadricoptero de arquitectura abierta con algunas modificaciones en el diseño que ayudarán a la carga y a la localizacion de los sensores de visión y localización. \cite{cite3}.

\subsection{Limitaciones del Estudio}

Este estudio se limita a la validación técnica del sistema en un entorno controlado y con un tipo de drone específico (cuadricóptero de arquitectura abierta). No se abordará la escalabilidad a otros tipos de drones ni su implementación en condiciones climáticas adversas. Además, el proyecto no incluye pruebas prolongadas ni la implementación en aplicaciones comerciales.

Las tecnologías de visión y localización implementadas también estarán limitadas a las condiciones de prueba, y su precisión en escenarios más complejos no será evaluada en este trabajo. El objetivo es demostrar la viabilidad técnica de los componentes clave, más que ofrecer una solución final completamente funcional \cite{cite4}.

\section{Estructura de la Tesis}

El capítulo 2 presentará una revisión detallada del estado del arte en sistemas de carga para drones abarcando temas de mecánica, electrónica y programación. El capítulo 3 describirá la metodología empleada para el desarrollo del prototipo. El capítulo 4 discutirá los resultados obtenidos durante las pruebas, y el capítulo 5 incluirá las conclusiones y recomendaciones para futuros trabajos.

 % Llama al archivo 'introduccion.tex' desde la carpeta 'chapters'

\chapter{Estado del Arte} % Capítulo de estado del arte
% Estado del Arte para la tesis
% Documento independiente para el estado del arte

\section{Estado del Arte}

El propósito de este capítulo es brindar una visión clara y detallada de los trabajos e investigaciones previas en el campo relacionado con la carga y aterrizaje de enjambres de drones. Se presenta una revisión de la literatura, las tecnologías existentes y las limitaciones identificadas que justifican la necesidad de la investigación propuesta.

\section{Programación}

\subsection{Visión por Computadora}

La visión por computadora se ha convertido en una herramienta clave para el aterrizaje preciso de drones. Investigaciones recientes han utilizado técnicas como el reconocimiento de patrones y redes neuronales para mejorar la precisión durante el aterrizaje \cite{Smith2020}. Estas técnicas permiten ajustar la posición del dron en tiempo real basándose en la detección visual de la plataforma de aterrizaje.

\subsection{Control de Drones}

\subsubsection{Control Manual}

El control manual sigue siendo común en aplicaciones donde se requiere una supervisión humana constante, como en escenarios de rescate. Aunque proporciona flexibilidad, limita la autonomía y la eficiencia del sistema.

\subsubsection{Control Automático}

El control automático de drones implica la integración de sensores y algoritmos de control que permiten al dron tomar decisiones en tiempo real. Investigaciones como \cite{Lee2021} han demostrado cómo los algoritmos de visión y la integración con GPS pueden mejorar significativamente la capacidad de los drones para aterrizar de manera autónoma y precisa.

\section{Diseño Eléctrico}

\subsection{Carga con Cable}

Las estaciones de carga por cable han sido ampliamente utilizadas debido a su simplicidad. Sin embargo, la necesidad de conexión física limita la autonomía y presenta desafíos en términos de desgaste y mantenimiento.

\subsection{Carga Inalámbrica}

La carga inalámbrica, utilizando resonancia magnética o inducción, ha sido propuesta como una solución para mejorar la eficiencia de los enjambres de drones. \cite{Shen2019} muestra cómo este método reduce la necesidad de intervención humana, aunque presenta limitaciones relacionadas con la alineación y la eficiencia de la transferencia de energía.

\subsection{Parámetros de Carga}

Es fundamental definir los parámetros de carga adecuados para asegurar la longevidad y eficiencia de las baterías de los drones. Los estudios destacan la importancia de controlar la temperatura y la corriente durante el proceso de carga para evitar daños \cite{Kumar2022}.

\section{Diseño Mecánico}

\subsection{Tipos de Estaciones de Carga}

Existen diferentes tipos de estaciones de carga para drones, desde las que requieren conexión física hasta las inalámbricas. Las estaciones automatizadas se están desarrollando para mejorar la eficiencia de carga y reducir la necesidad de intervención humana \cite{Lopez2023}.

\subsection{Simulaciones de Ingeniería}

Las simulaciones de ingeniería son esenciales para validar el diseño y la operación de las estaciones de carga antes de su construcción. Herramientas como MATLAB y ANSYS se utilizan para modelar las fuerzas y las tensiones involucradas durante el aterrizaje y la carga.

\subsection{Dibujos de Ingeniería}

Los dibujos de ingeniería detallan las especificaciones del diseño de la estación de carga, incluyendo dimensiones, materiales y componentes. Estos planos son fundamentales para garantizar que el diseño sea fabricable y cumpla con los requisitos de seguridad y eficiencia.


 % Llama al archivo 'estado_arte.tex' desde la carpeta 'chapters'

\chapter{Desarrollo} % Capítulo principal de desarrollo
% Desarrollo

Este capítulo describe el proceso de diseño, implementación y configuración de la estación de carga modular y la instrumentación del dron. Se detallan las metodologías y herramientas utilizadas para llevar a cabo el desarrollo, con un enfoque en la creación de una solución eficaz para el sistema de carga.

\section{Diseño y Desarrollo de la Base de Carga}
\subsection{Especificaciones y Requerimientos}
A continuación se presentan las especificaciones técnicas y los requisitos para la base de carga, garantizando compatibilidad y eficiencia para un sistema de carga en enjambres de drones:
    \begin{enumerate}
        \item Diseño modular y apilable para facilitar el almacenamiento de múltiples drones en un mismo metro cuadrado.
        \item Las dimensiones mínimas requeridas para la estacion de carga debe ser de mínimo 10 porciento mayores en ancho, largo y alto que las del drone.
        \item La estacion de carga deberá contar con un mecanismo de carga para la recarga de la batería del dron.
        \item Debe contar con por lo menos un método de posicionamiento visible para el drone.
    \end{enumerate}

\subsection{Selección e Implementación del Mecanismo de Movimiento del Cajón}

En la sección 2.2.3 se explican en detalle los diferentes mecanismos de movimiento que se consideraron para la estación de carga, evaluando sus ventajas y desventajas en términos de precisión, durabilidad, costo y facilidad de integración en el diseño modular. Se analizaron opciones como pistones lineales, bandas dentadas y otros sistemas de transmisión, con el objetivo de seleccionar una solución que cumpliera con los requisitos de funcionalidad sin elevar demasiado los costos. 
Tras esta evaluación, se optó por un mecanismo de Polea y Banda tipo V, el cual, además de ser una opción económica, utiliza componentes ya disponibles, como el motor. Este sistema permite un movimiento horizontal suficiente para posicionar el dron con la precisión necesaria para el proceso de carga, a la vez que facilita el apilamiento modular de la estación. La orientación del movimiento no obstruye la estructura general de la estación, lo cual asegura que el diseño cumpla con el requisito de apilabilidad y compatibilidad en espacios compartidos, manteniéndose compacto y adaptable sin comprometer la estructura modular de la estación.

Para este mecanismo, se seleccionaron diferentes componentes, incluyendo una banda de alta resistencia para soportar las cargas previstas, poleas (con baja fricción) para optimizar el movimiento y reducir el desgaste del sistema en operaciones continuas, y un motor ... que tiene un torque de ... y cumple con el torque necesario para mover los 3 kg de peso del drone.

\subsubsection{Diseños CAD y Componentes}
    \begin{enumerate}
        \item Motor de 12V
        \item Poleas
        \item Banda
        \item Baleros
        \item Chumaceras 
        \item Soporte del Motor
    \end{enumerate}

\subsubsection{Mecanismo Polea y Banda tipo V}
(Imagen del Mecanismo)

\subsection{Diseño CAD de la Base de Carga}
Para cumplir con los requerimientos de diseño modular / apilable y compatibilidad con el drone en cuestión a las dimensiones de este. Se llevó a cabo el siguiente diseño en CAD:

    \begin{itemize}
        \item     El diseño de la estructura principal de la estación de carga se realizó en Fusion 360, considerando las dimensiones mínimas requeridas para el dron y el mecanismo de carga.
        Imagen
        \item     Se tomó en cuenta la dimensión necesaria para que el drone tenga un espacio de por lo menos 10 cm de separacion con la pared del cajón, esto para evitar alguna colusión del drone con la estructura de la base de carga al momento de intentar aterrizar, además se disminuyú la altura del cajón para que el drone pueda aterrizar sin problemas. Por otro lado se usarón rieles que se obtuvieron previamente reutilizados, por lo cual la estacion de carga física se encuentra salida de la base de carga por unos 5 cm al momento de estar cerrada, pero en el diseño CAD el diseño si se encuentra cerrado por completo. 
        Imagen
        \item     Se añadió un espacio de 25 cm para colocar la parte electronica del cajón.
        imagen
    \end{itemize}

\subsection{Lista de Materiales}
\begin{itemize}
    \item Perfiles de Aluminio
        \begin{itemize}
            \item 2 x Perfiles de 75 cm
            \item 2 x Perfiles de 50 cm
            \item 2 x Perfiles de 25 cm
        \end{itemize}
    \item 1 x Fuente de Poder de 12V
    \item 1 x Arduino Mega
    \item 1 x Raspberry Pi 4
    \item 1 x Driver JBT 
    \item 2 x Sensor de Proximidad Inductivo
    \item 1 x Joystick
    \item 1 x Cargador de Lipo battery
    \item Tronillería para perfil de aluminio
    \item 2 x Rieles de 75 cmd
    \item 1 x Motor de 12V
    \item 1 x Soporte de Motor impreso en 3D
    \item 1 x Banda de 1 m
    \item 2 x Poleas de 5 cm de diámetro
    \item 2 x Baleros de 5 cm de diámetro
    \item 2 x Chumacera de 5 cm de diámetro impresas en 3D
    \item 2 x MDF de 9 mm de 50 x 50 cm
    \item 8 x Separadores de MDF impresos en 3D
    \item 1 x Router WiFi
    \item 16 x Codos para perfil de aluminio impresos en 3D
\end{itemize}

\subsection{Proceso de Manufactura de la Estructura de la Base de Carga}

    \begin{enumerate}
        \item \textbf{Corte del Material:} Para iniciar el proceso de manufactura, se realizaron cortes precisos de los perfiles de aluminio y las placas de MDF según las dimensiones especificadas en la lista de materiales. Este paso es fundamental para asegurar que todas las piezas se ensamblen correctamente en el diseño modular.
            \begin{center}
                \textit{Imagen del proceso de corte de perfiles de aluminio y MDF}
            \end{center}
        \item \textbf{Ensamblaje de la Estructura:} Una vez cortadas las piezas, se procedió a ensamblar la estructura principal de la estación de carga utilizando tornillos y tuercas para fijar los perfiles de aluminio y las placas de MDF usando los separadores impresos. Se verificó que todas las piezas estuvieran alineadas y niveladas para garantizar la estabilidad y resistencia de la base de carga.
            \begin{center}
                \textit{Imagen del proceso de ensamblaje de la estructura de la base de carga}
            \end{center}
        \item \textbf{Instalación del Mecanismo de Movimiento:} Después de ensamblar la estructura principal, se instaló el mecanismo de polea y banda tipo V para permitir el movimiento horizontal del cajón. Se colocaron las poleas, la banda y el motor en las ubicaciones previamente definidas, asegurando que el sistema de movimiento funcionara correctamente y sin obstrucciones.
            \begin{center}
                \textit{Imagen del proceso de instalación del mecanismo de movimiento}
            \end{center}
        
    \end{enumerate}



\subsection{Circuito Electrónico de la Estación de Carga}

    \subsubsection{Diagrama de Conexiones}
    Añadir diagrama de conexiones

    \subsubsection{Contrucción del Circuito}
    Añadir imágenes de la construcción del circuito

    \subsubsection{Programación del Circuito}
    Explicar el código de programación del circuito y su funcionamiento, mencionar que en los anexos se encuentra el código fuente del arduino y raspberry pi 4.


\section{Instrumentación del Dron}

\subsection{Especificaciones y Requerimientos}
Para garantizar la compatibilidad y funcionalidad en el sistema de carga, el dron debe cumplir con las siguientes especificaciones:
    \begin{itemize}
        \item Las dimensiones de las piezas deberán adaptarse al frame del drone cuadricoptero de arquitectura abierta que fue comprado.
        \item El dron deberá tener un circuito de carga compatible con la estacion de carga.
        \item El drone deberá contar con una cámara y un sistema de visión para la detección de marcadores Aruco.
        \item Se deberán integrar sensores de localización para la navegación en exterior.
        \item Se deberá integrar algún microprocesador como computadora auxiliar para el procesamiento de datos y la comunicación con la estación de carga.
    \end{itemize}

\subsection{Lista de Materiales para la Instrumentación del Dron} 
    \begin{itemize} 
        \item 1 x Frame de cuadricóptero abierto (especificar modelo) 
        \item 1 x Controlador de vuelo (especificar modelo) 
        \item 1 x Cámara (compatible con detección Aruco) 
        \item 1 x Raspberry Pi 4 (Companion Computer) 
        \item 1 x Sensor de proximidad (modelo de preferencia) 
        \item 1 x Módulo GPS (compatible con el controlador de vuelo) 
        \item Cableado y conectores 
        \item Material de montaje impreso en 3D (soportes específicos para cada componente) 
    \end{itemize}

\subsection{Diseños CAD del Dron} 
Se presentan a continuación los diseños CAD de las piezas que se han desarrollado para el dron, adaptadas a su frame original para integrar los componentes electrónicos necesarios y cumplir con los requisitos de carga y posicionamiento.

    \begin{itemize} 
        \item Se diseñaron nuevos soportes para el microprocesador y la cámara, asegurando una integración estable y precisa en el frame. 
        \item Se implementó un espacio de montaje para los sensores de localización y el circuito de carga. 
        \begin{center} 
            \textit{Imagen de los diseños CAD del dron con las piezas modificadas} 
        \end{center} 
    \end{itemize}



\subsection{Circuito de Distribución de Energía} 
El circuito de distribución de energía proporciona alimentación segura y estable a los componentes del dron. La selección de baterías y reguladores de voltaje fue optimizada para asegurar la duración de vuelo y protección de cada componente.

    \begin{itemize} 
        \item Se seleccionaron baterías de litio-polímero (LiPo) de alta capacidad para maximizar la autonomía. 
        \item Reguladores de voltaje se añadieron para adaptar el suministro a la Raspberry Pi, la cámara y el controlador de vuelo. 
        \begin{center} 
            \textit{Imagen del circuito de distribución de energía en el dron} 
        \end{center} 
    \end{itemize}

    \subsection{Integración de Componentes Electrónicos} Cada componente fue ensamblado y cableado de acuerdo con el diseño modular del dron. A continuación se describen los pasos del proceso de integración:

    \begin{enumerate} 
        \item \textbf{Montaje de la Cámara:} La cámara se fijó en la parte de abajo del dron, permitiendo una visibilidad óptima para la detección de marcadores Aruco. 
            \begin{center} 
                \textit{Imagen del montaje de la cámara en el dron} 
            \end{center} 
        \item \textbf{Instalación del Controlador de Vuelo y Módulo GPS:} El controlador de vuelo se instaló en el centro del frame para optimizar el balance y calibración, y el módulo GPS se colocó en una de las patas. 
            \begin{center} 
                \textit{Imagen del montaje del controlador de vuelo y módulo GPS} 
            \end{center} 
        \item \textbf{Instalación del Microprocesador:} La Raspberry Pi 4 se integró en un soporte impreso en 3D en la parte superior, y el microprocesador se colocó sobre este soporte. Un cable de comunicación serial se conectó entre la Raspberry Pi y el controlador de vuelo que se encuentra debajo de el microprocesador.
            \begin{center} 
                \textit{Imagen del montaje del microprocesador en el dron} 
            \end{center} 
    \end{enumerate}

\section{Software Configuration}

\subsection{Configuración de la Computadora Central en Tierra} 
La computadora central se configuró para supervisar y procesar la información proveniente de la computadora auxiliar del dron. A continuación, se describen los pasos realizados para esta configuración:

\begin{itemize}
    \item \textbf{Instalación de ROS 2 Humble:} 
    Dado que la computadora central utiliza Ubuntu 22.04, se instaló ROS 2 Humble siguiendo las instrucciones oficiales. Los comandos utilizados fueron:
    \begin{verbatim}
    sudo apt update && sudo apt install -y software-properties-common
    sudo add-apt-repository universe
    sudo apt update
    sudo apt install -y ros-humble-desktop
    \end{verbatim}
    Esto incluyó la instalación de las herramientas necesarias para desarrollar y ejecutar aplicaciones en ROS 2 Humble.

    \item \textbf{Configuración del entorno:} 
    Para facilitar el uso de ROS 2, se configuró el archivo \texttt{~/.bashrc}. Se agregó la siguiente línea al final del archivo:
    \begin{verbatim}
    source /opt/ros/humble/setup.bash
    \end{verbatim}
    Posteriormente, se ejecutó:
    \begin{verbatim}
    source ~/.bashrc
    \end{verbatim}

    \item \textbf{Clonación del repositorio de GitHub:} 
    Se creó un espacio de trabajo para ROS 2 y se clonó el repositorio correspondiente. Los pasos realizados fueron:
    \begin{verbatim}
    mkdir -p ~/ros2_ws/src
    cd ~/ros2_ws/src
    git clone <URL_del_repositorio>
    cd ..
    colcon build
    \end{verbatim}

    \item \textbf{Configuración del \texttt{ROS\_DOMAIN\_ID}:} 
    Para garantizar una correcta comunicación entre la computadora central y la computadora auxiliar del dron, se configuró el \texttt{ROS\_DOMAIN\_ID} con el valor 10. Esto se realizó añadiendo la siguiente línea al archivo \texttt{~/.bashrc}:
    \begin{verbatim}
    export ROS_DOMAIN_ID=10
    \end{verbatim}
    Luego, se ejecutó:
    \begin{verbatim}
    source ~/.bashrc
    \end{verbatim}
    
    \item \textbf{Verificación del entorno de ROS 2:} 
    Finalmente, se verificó que el entorno estuviera correctamente configurado utilizando los siguientes comandos:
    \begin{verbatim}
    ros2 doctor
    \end{verbatim}
    Esto permitió confirmar que todas las dependencias necesarias para ROS 2 Humble estuvieran instaladas y funcionando correctamente.
    
    \begin{center} 
        \textit{Imagen de la configuración del entorno en la computadora central.} 
    \end{center}
\end{itemize}


\subsection{Configuración de la Computadora Auxiliar del Dron} 
    La Raspberry Pi 5 se configuró como computadora auxiliar para el procesamiento de datos y comunicación en tiempo real con la estación de carga. A continuación, se detallan los pasos realizados para su configuración:
    
    \begin{itemize}
        \item \textbf{Instalación de Ubuntu 24.04:} 
        Se utilizó la herramienta Raspberry Pi Imager para instalar Ubuntu Server 24.04 LTS (64-Bit) en la tarjeta SD. Durante la configuración inicial, se habilitó SSH, se definió un usuario con contraseña y se conectó la Raspberry Pi a la red WiFi.
    
        \item \textbf{Conexión inicial y SSH:} 
        Después de insertar la tarjeta SD en la Raspberry Pi y conectarla a un monitor y teclado, se encendió el dispositivo e ingresaron las credenciales configuradas. Se obtuvo la dirección IP mediante el comando:
        \begin{verbatim}
        hostname -I
        \end{verbatim}
        Con esta información, se estableció una conexión SSH desde una computadora externa utilizando:
        \begin{verbatim}
        ssh <usuario>@<IP>
        \end{verbatim}
        Esto permitió continuar con la configuración de la Raspberry Pi de forma remota.
    
        \item \textbf{Actualización del sistema:} 
        Se realizó una actualización completa del sistema operativo con los siguientes comandos:
        \begin{verbatim}
        sudo apt update && sudo apt upgrade -y
        \end{verbatim}
        También se instaló \texttt{raspi-config} para habilitar el puerto serial. Esta opción se configuró en \texttt{Interface Options > Serial Port}, seleccionando \texttt{No} y luego \texttt{Yes}.
    
        \item \textbf{Modificación de \texttt{bashrc} y configuración del ID:} 
        Para garantizar la comunicación entre todos los dispositivos, se configuró el \texttt{ROS\_DOMAIN\_ID} con el valor 10. Se editó el archivo \texttt{~/.bashrc} con:
        \begin{verbatim}
        sudo vim ~/.bashrc
        \end{verbatim}
        Al final del archivo, se agregaron las siguientes líneas:
        \begin{verbatim}
        export ROS_DOMAIN_ID=10
        source /opt/ros/jazzy/setup.bash
        \end{verbatim}
        Posteriormente, se guardaron los cambios y se ejecutó:
        \begin{verbatim}
        source ~/.bashrc
        \end{verbatim}
    
        \item \textbf{Instalación de ROS 2 Jazzy:} 
        Se siguieron las instrucciones oficiales para instalar ROS 2 Jazzy en Ubuntu 24.04. Esto incluyó los comandos:
        \begin{verbatim}
        sudo apt install -y software-properties-common
        sudo add-apt-repository universe
        sudo apt update
        sudo apt install -y ros-jazzy-desktop
        \end{verbatim}
    
        \item \textbf{Clonación del repositorio de GitHub:} 
        Se creó un espacio de trabajo en ROS 2 para integrar los nodos personalizados. Los pasos realizados fueron:
        \begin{verbatim}
        mkdir -p ~/ros2_ws/src
        cd ~/ros2_ws/src
        git clone <URL_del_repositorio>
        cd ..
        colcon build
        \end{verbatim}
    
        \item \textbf{Instalación de MAVROS y MAVProxy:} 
        Para la comunicación con el Pixhawk, se instalaron MAVROS y MAVProxy. Los comandos utilizados fueron:
        \begin{verbatim}
        sudo apt install ros-jazzy-mavros ros-jazzy-mavros-extras
        sudo rosdep init
        rosdep update
        sudo apt install python3-mavproxy
        \end{verbatim}
        Además, se configuraron los complementos geográficos necesarios:
        \begin{verbatim}
        sudo apt install geographiclib-tools
        sudo geographiclib-get-geoids egm96-5
        \end{verbatim}
    
        \item \textbf{Verificación de la comunicación:} 
        Para garantizar que MAVROS y MAVProxy funcionaran correctamente, primero se inició MAVProxy con:
        \begin{verbatim}
        mavproxy.py --master=/dev/ttyAMA0 --baudrate 921600
        \end{verbatim}
        Posteriormente, se lanzó MAVROS utilizando:
        \begin{verbatim}
        ros2 launch mavros px4.launch fcu_url:=serial:///dev/ttyAMA0:921600
        \end{verbatim}
        Finalmente, se verificaron los tópicos disponibles con:
        \begin{verbatim}
        ros2 topic list
        \end{verbatim}
        y se confirmó la comunicación observando los mensajes de los tópicos relevantes.
        
        \begin{center} 
            \textit{Imagen de la configuración del entorno en la Raspberry Pi 5.} 
        \end{center}
    \end{itemize}
    
\subsection{Configuración de la Estación de Control en Tierra} 
    Se detalla la configuración realizada para la estación de control en tierra utilizando las herramientas QGroundControl y Mission Planner. Ambas aplicaciones son similares en funcionalidad, permitiendo la visualización y gestión de parámetros de vuelo. Sin embargo, dado que se utilizó ArduPilot como firmware, se decidió priorizar Mission Planner debido a su optimización para este sistema.
    
    \begin{itemize}
        \item \textbf{Instalación de Mission Planner y QGroundControl:} 
        Se instalaron ambas herramientas en la computadora central para la gestión y monitoreo del Pixhawk. Mission Planner se utilizó principalmente por su compatibilidad directa con ArduPilot, mientras que QGroundControl fue útil en ciertas configuraciones iniciales.
    
        \item \textbf{Calibración de sensores:} 
        La calibración de los sensores del Pixhawk se realizó desde Mission Planner, siguiendo estos pasos:
        \begin{itemize}
            \item Acceder a la sección de calibración en la pestaña \textit{Initial Setup}.
            \item Seleccionar la opción de calibración para cada sensor:
            \begin{itemize}
                \item \textbf{Acelerómetro:} Se colocó el Pixhawk en diferentes orientaciones según las instrucciones en pantalla para calibrar correctamente.
                \item \textbf{Giroscopio:} Se mantuvo el Pixhawk inmóvil durante el proceso de calibración.
                \item \textbf{GPS:} Se verificó la recepción de satélites y se ajustaron los parámetros necesarios para una correcta ubicación.
            \end{itemize}
        \end{itemize}
        \begin{center} 
            \textit{Imagen de la configuración de sensores en Mission Planner.} 
        \end{center}
    
        \item \textbf{Parámetros de comunicación:} 
        Para establecer la comunicación entre la Raspberry Pi y el Pixhawk a través de MAVROS y MAVLink, se realizaron los siguientes ajustes en los parámetros utilizando Mission Planner:
        \begin{itemize}
            \item \texttt{SERIAL2\_PROTOCOL}: Configurado en 2 para habilitar MAVLink.
            \item \texttt{SERIAL2\_BAUD}: Ajustado a 921600 para coincidir con la configuración de la Raspberry Pi.
            \item \texttt{SYS\_ID}: Configurado en el ID correspondiente para garantizar una comunicación adecuada con el sistema.
        \end{itemize}
    
        \item \textbf{Validación de la configuración:} 
        Se verificó la conexión entre la Raspberry Pi y el Pixhawk mediante el comando:
        \begin{verbatim}
        ros2 topic list
        \end{verbatim}
        Además, se confirmaron los mensajes de MAVLink desde Mission Planner utilizando la consola de mensajes.
    
    \end{itemize}
    

\section{Sistema de Visión}
\subsection{Especificaciones y Requerimientos} 
El sistema de visión debe cumplir con los siguientes requisitos para asegurar la precisión en la detección de marcadores Aruco:
    \begin{itemize}
        \item Obtención y aplicación de las matrices de calibración de la cámara.
        \item Generar marcadores Aruco de diferentes tamaños (Embedded ArUcos) para su detección en tiempo real.
    \end{itemize}

\subsection{Calibración de la Cámara} El proceso de calibración incluyó la captura de múltiples imágenes de una cuadrícula de calibración y la generación de los parámetros intrínsecos y extrínsecos.

    \begin{center} 
        % Imagen del proceso de calibración y los resultados obtenidos
        \figurename{
            \begin{figure}
                \centering
                %\includegraphics[width=0.8\textwidth]{pictures/calibracion_camara.png}
                \caption{Imagen del proceso de calibración y los resultados obtenidos}
            \end{figure}
        }
        
        \textit{Imagen del proceso de calibración y los resultados obtenidos} 
    \end{center}

\subsection{Generación de Marcadores Aruco} 
Se generaron marcadores Aruco adaptados al sistema, incluyendo tamaños y patrones específicos para maximizar la precisión en la detección desde diferentes alturas.
    \begin{center} 
        \textit{Imagen de los diferentes tamaños y diseños de los marcadores Aruco} 
    \end{center}

\subsection{Detección de Marcadores e-Aruco en Tiempo Real} 
El proceso que se llevó a cabo para la detección de marcadores Aruco en tiempo real se describe a continuación:

    \begin{enumerate} 
        \item \textbf{Captura de Imágenes:} Se capturaron imágenes de la cámara en tiempo real, utilizando la webcam concectada a la Raspberry Pi 5 que se encuentra en el drone. 
        \item \textbf{Procesamiento de Pose de los ArUcos:} Se aplicó un algoritmo para la deteccion de la pose de los marcadores ArUco el cual se describe a continuación: 
            \begin{itemize} 
                \item Se utilizó la librería OpenCV para el procesamiento de imágenes y la detección de marcadores Aruco. 
                \item Se implementó un algoritmo de detección de bordes y esquinas para identificar los marcadores en las imágenes. 
                \item Se mide el centroide de los marcadores y se calcula la matriz de posición y orientación para determinar la ubicación del drone.
             \end{itemize}
        
        \item \textbf{Visualización en Tiempo Real:} Se mostraron los marcadores detectados en tiempo real, permitiendo la interacción con el sistema de control en tierra.
    \end{enumerate}

    \begin{center} 
        \textit{Imagen de la detección de marcadores Aruco en tiempo real} 
    \end{center}

\section{Sistema de Comunicación}
\subsection{Especificaciones y Requerimientos} 
Para asegurar una comunicación efectiva entre el dron y la estación de carga, el sistema de comunicación debe cumplir con los siguientes requisitos:

    \begin{itemize} 
        \item Utilización de un protocolo basado en \textbf{WiFi} para la transferencia de datos entre el dron y la estación de carga. 
        \item Implementación del middleware DDS (Data Distribution Service) de ROS 2 para publicación y suscripción de tópicos en tiempo real. 
    \end{itemize}

\subsection{Protocolo de Comunicación entre el Dron y la Base de Carga} 
El protocolo de comunicación establece la interacción entre el dron y la estación de carga, utilizando ROS 2 y DDS para la transmisión de mensajes en tiempo real.

    \begin{center} 
        \textit{Diagrama del flujo de comunicación entre el dron y la base de carga} 
    \end{center}

\subsubsection{Diagrama de Comunicación WiFi} 
Este diagrama muestra la estructura de comunicación dentro de la red WiFi, destacando el flujo de datos desde el dron hasta la estación de carga.

\subsubsection{Diagrama de Comunicación de Nodos ROS 2} 
El esquema de nodos en ROS 2 especifica los tópicos y servicios clave utilizados para la comunicación entre el dron y la estación de carga.

    \begin{center} 
        \textit{Diagrama de comunicación de nodos en ROS 2} 
    \end{center}




 % Llama al archivo 'desarrollo.tex' desde la carpeta 'chapters'

\chapter{Conclusiones} % Capítulo de conclusiones

En esta sección, se concluye todo lo que se realizó durante el desarrollo de la tesis, desde el diseño y la fabricación de la estación de carga modular hasta la instrumentación del dron con sensores de localización y un sistema de visión por computadora. Se logró cumplir con los objetivos planteados, destacando la modularidad y eficiencia del sistema de carga propuesto, así como la integración exitosa del dron con la estación de carga.

A pesar de los logros alcanzados, se identificaron áreas de mejora que podrían explorarse en trabajos futuros. Entre estas mejoras se encuentra la implementación de un sistema de carga rápida, que permitiría reducir significativamente el tiempo de carga del dron. Asimismo, sería beneficioso desarrollar un sistema de iluminación autónoma para garantizar un aterrizaje preciso en condiciones de poca luz, mejorando así la autonomía operativa del dron en diferentes escenarios.

Finalmente, la optimización del sistema de comunicación entre el dron y la estación de carga podría incrementar la eficiencia del proceso de aterrizaje y carga, especialmente cuando se opera con múltiples estaciones en un entorno de enjambre. % Llama al archivo 'conclusiones.tex' desde la carpeta 'chapters'

% Bibliografía
\begin{thebibliography}{9}
    \bibitem{ejemplo} Autor, ``Título del artículo o libro'', Año. % Ejemplo de referencia bibliográfica
\end{thebibliography}

\end{document}