% Estado del Arte para la tesis
% Documento independiente para el estado del arte

\section{Estado del Arte}

El propósito de este capítulo es brindar una visión clara y detallada de los trabajos e investigaciones previas en el campo relacionado con la carga y aterrizaje de enjambres de drones. Se presenta una revisión de la literatura, las tecnologías existentes y las limitaciones identificadas que justifican la necesidad de la investigación propuesta.

\section{Programación}

\subsection{Visión por Computadora}

La visión por computadora se ha convertido en una herramienta clave para el aterrizaje preciso de drones. Investigaciones recientes han utilizado técnicas como el reconocimiento de patrones y redes neuronales para mejorar la precisión durante el aterrizaje \cite{Smith2020}. Estas técnicas permiten ajustar la posición del dron en tiempo real basándose en la detección visual de la plataforma de aterrizaje.

\subsection{Control de Drones}

\subsubsection{Control Manual}

El control manual sigue siendo común en aplicaciones donde se requiere una supervisión humana constante, como en escenarios de rescate. Aunque proporciona flexibilidad, limita la autonomía y la eficiencia del sistema.

\subsubsection{Control Automático}

El control automático de drones implica la integración de sensores y algoritmos de control que permiten al dron tomar decisiones en tiempo real. Investigaciones como \cite{Lee2021} han demostrado cómo los algoritmos de visión y la integración con GPS pueden mejorar significativamente la capacidad de los drones para aterrizar de manera autónoma y precisa.

\section{Diseño Eléctrico}

\subsection{Carga con Cable}

Las estaciones de carga por cable han sido ampliamente utilizadas debido a su simplicidad. Sin embargo, la necesidad de conexión física limita la autonomía y presenta desafíos en términos de desgaste y mantenimiento.

\subsection{Carga Inalámbrica}

La carga inalámbrica, utilizando resonancia magnética o inducción, ha sido propuesta como una solución para mejorar la eficiencia de los enjambres de drones. \cite{Shen2019} muestra cómo este método reduce la necesidad de intervención humana, aunque presenta limitaciones relacionadas con la alineación y la eficiencia de la transferencia de energía.

\subsection{Parámetros de Carga}

Es fundamental definir los parámetros de carga adecuados para asegurar la longevidad y eficiencia de las baterías de los drones. Los estudios destacan la importancia de controlar la temperatura y la corriente durante el proceso de carga para evitar daños \cite{Kumar2022}.

\section{Diseño Mecánico}

\subsection{Tipos de Estaciones de Carga}

Existen diferentes tipos de estaciones de carga para drones, desde las que requieren conexión física hasta las inalámbricas. Las estaciones automatizadas se están desarrollando para mejorar la eficiencia de carga y reducir la necesidad de intervención humana \cite{Lopez2023}.

\subsection{Simulaciones de Ingeniería}

Las simulaciones de ingeniería son esenciales para validar el diseño y la operación de las estaciones de carga antes de su construcción. Herramientas como MATLAB y ANSYS se utilizan para modelar las fuerzas y las tensiones involucradas durante el aterrizaje y la carga.

\subsection{Dibujos de Ingeniería}

Los dibujos de ingeniería detallan las especificaciones del diseño de la estación de carga, incluyendo dimensiones, materiales y componentes. Estos planos son fundamentales para garantizar que el diseño sea fabricable y cumpla con los requisitos de seguridad y eficiencia.


