% Description: This file contains the hardware state of the art.

\section{Hardware}

\subsection{Drones}
    \subsubsection{Types of Drones}
    Drones available in the market vary widely based on their intended applications, features, and prices. The following are the main types of drones that can be found:
    
    \paragraph{Consumer Drones} These drones are intended for recreational use and typically come equipped with a camera for capturing photos and videos. They are relatively affordable, with prices ranging from \$200 to \$1,500 depending on their features. Popular models include DJI's Phantom and Mavic series.
    
    \paragraph{Commercial Drones} These drones are designed for specific commercial purposes, such as surveying, mapping, inspection, and agriculture. They usually feature specialized sensors, like multispectral cameras, and have advanced software for planning and analyzing missions. Their price can range from \$2,000 to \$20,000. Examples include DJI's Matrice series and Parrot's Anafi.
    
    \paragraph{Industrial Drones} Industrial drones are high-performance devices used in demanding applications, such as power line inspections, oil and gas facility monitoring, and cargo transport. These drones are highly customizable and have long flight times, strong payload capabilities, and robust structures. Prices vary widely but generally exceed \$20,000, depending on the customization and capabilities.
    
    \paragraph{Racing Drones} Racing drones are built for speed, agility, and performance in drone racing competitions. They are lightweight, feature powerful motors, and are capable of speeds up to 150 km/h. Prices start around \$300 and can exceed \$2,000 for high-end models used by professional racers.
    
    \paragraph{Military Drones} Military drones, also known as Unmanned Aerial Vehicles (UAVs), are used for surveillance, reconnaissance, and combat missions. These drones are the most sophisticated, featuring stealth capabilities, long endurance, and powerful payloads. Costs vary significantly, often running into millions of dollars.
    
    The type of drone chosen depends on the specific requirements of the task, such as the payload capacity, flight time, range, and available budget.

\subsection{Charging Stations}
    \subsubsection{Types of Charging Stations}
    There are several types of charging stations for drones currently available in the market, each designed to meet specific needs depending on the type of drone, its intended application, and operational environment. The most common types of drone charging stations include:
    
    \paragraph{Contact-Based Charging Stations} These are among the most widely used solutions in the market, typically relying on metal contacts between the drone and the station to transfer electrical energy. Contact-based charging stations are relatively inexpensive and are suited for various commercial applications, including surveying, surveillance, and agriculture. They require precise landing, as proper alignment is necessary to establish the connection.
    
    \paragraph{Wireless Charging Stations} Wireless charging, also known as inductive charging, uses electromagnetic fields to transfer power to the drone without physical contact. While this technology provides more flexibility in terms of landing, it is less efficient compared to contact-based charging and has higher production costs. It is often utilized in specialized environments where drones must be kept operational without manual intervention, such as security monitoring or in areas where contact charging is impractical.
    
    \paragraph{Battery Swap Stations} These stations are designed to swap out depleted drone batteries with fully charged ones. The process is automated, allowing for continuous drone operation with minimal downtime. This approach is typically used in high-demand applications, such as drone delivery services and industrial inspections, where the quick turnaround is crucial.
    
    \paragraph{Solar-Powered Charging Stations} Solar-powered stations use solar panels to generate energy, which is stored and used to charge the drone. They are especially popular in remote areas where there is no direct access to power grids. These stations are environmentally friendly and allow drones to operate in off-grid locations, making them ideal for environmental monitoring and research.
    
    Overall, the choice of charging station depends on the intended application of the drone, the available infrastructure, and the required operational efficiency.

\subsection{Automatic Drawer Mechanism}
    \subsubsection{Types of Mechanisms}
    The automatic drawer mechanism in the charging station for drone storage is similar to the mechanisms found in other automatic storage solutions. Here, we'll discuss the main mechanisms that could be used in this context, including the advantages and disadvantages of each. The mechanism chosen for this project was the belt and V-pulley due to its easy accessibility and low cost.
    
    \paragraph{Pistons} Pistons are a common choice for automated movement in many types of machinery. They use compressed air or hydraulic fluid to extend and retract, making them well-suited for precise and strong movements. For an automatic drawer for drones, pistons could provide a robust and stable means to open and close the drawer. However, this approach may have higher costs and complexity compared to other mechanisms, as it requires pneumatic or hydraulic infrastructure and regular maintenance.
    
    \paragraph{Belt and Toothed Pulley} This mechanism uses a toothed belt and pulley to move the drawer. It offers greater precision compared to V-belts because the toothed design prevents slipping, making it a suitable choice for environments where precise positioning is critical. However, the increased complexity and cost of toothed belts make them less accessible, and they may require more maintenance due to the increased friction between the teeth.
    
    \paragraph{Belt and V-Pulley} This mechanism involves a belt shaped like a V and a pulley, which ensures good friction and prevents slipping while moving the drawer. The belt and V-pulley mechanism is simple, inexpensive, and easy to source, which made it the choice for this project. It offers a balance between precision and accessibility, being a practical solution for the needs of an automatic drone storage drawer. Its main disadvantage is that it may lack the precise positioning capabilities of toothed pulleys, but for the current application, this is acceptable.

\subsection{Motors and Propellers}
    \subsubsection{Motor Selection}
    The selection of motors and propellers for a drone is based primarily on the total weight that the drone must carry, which determines the thrust required. Brushless motors are commonly used in drones due to their efficiency, reliability, and power-to-weight ratio. The specific motors chosen for this project are A2212 1000KV Brushless Motors, which are popular for their balance between power and efficiency.
    
    The thrust required for a drone must be approximately double the total weight of the drone to ensure stable flight. For a drone weighing 2 kg, the motors must provide at least 4 kg of thrust in total. The A2212 1000KV motors are capable of providing sufficient thrust when paired with appropriate propellers, making them suitable for small- to medium-sized drones. The propellers should be chosen based on the motor's specifications, balancing thrust, efficiency, and maneuverability.
    
    \paragraph{Comparison with Other Motors}
    	extbf{2216 900KV Motors}: These motors are slightly larger than the A2212 and provide more torque, making them suitable for heavier drones or those needing more stability. They are often used with larger propellers, which allows for increased thrust but decreases agility.
    
    	extbf{1806 2300KV Motors}: These motors are designed for smaller, lighter drones or racing drones, providing higher RPMs and increased speed. They are not suitable for drones carrying heavy payloads, as they prioritize speed over thrust.
    
    The choice of motor is ultimately determined by the drone's intended purpose. The A2212 1000KV motors offer a good compromise for general use, providing enough thrust for stability while maintaining efficiency.

    
    


