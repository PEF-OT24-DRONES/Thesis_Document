% Introducción para la tesis
% Documento independiente para la introducción

\section{Contexto General y Motivación}

\subsection{Contexto del Problema}

Los enjambres de drones enfrentan desafíos significativos en términos de eficiencia de carga y aterrizaje automático preciso. Actualmente, las estaciones de carga existentes no permiten la carga de múltiples drones y no pueden asegurar un aterrizaje en una posición exacta, lo cual limita su autonomía y funcionalidad.

\subsection{Importancia del Problema}

Este problema es relevante debido a la creciente adopción de drones en diversas industrias, como la vigilancia, la entrega de mercancías, y la agricultura de precisión. La falta de estaciones de carga eficientes y precisas limita su uso continuo y autónomo, requiriendo intervención humana constante, lo cual reduce la eficiencia operativa y aumenta los costos.

\section{Estado del Arte}

\subsection{Revisión Breve del Estado del Arte}

Investigaciones recientes han propuesto el uso de transferencia inalámbrica de energía y métodos avanzados de carga, como los basados en visión e infrarrojos, para abordar los desafíos de la carga eficiente y el aterrizaje preciso de drones \cite{Gupta2022}. Estos avances buscan mejorar la autonomía de los drones y reducir la necesidad de intervención humana.

\subsection{Limitaciones de las Soluciones Actuales}

Las soluciones actuales presentan limitaciones significativas, como la falta de modularidad y la incapacidad de cargar múltiples drones de manera simultánea. Además, la precisión durante el aterrizaje sigue siendo un reto, especialmente en entornos dinámicos o poco estructurados \cite{Mourgelas2020}.

\section{Planteamiento del Problema}

\subsection{Definición Clara del Problema}

El problema principal radica en la falta de estaciones de carga modulares y autónomas que aseguren un aterrizaje preciso y la carga eficiente de múltiples drones. Este trabajo busca desarrollar una solución que permita la carga simultánea y segura de múltiples drones sin intervención humana.

\subsection{Preguntas de Investigación}

- ¿Cómo se puede desarrollar una estación de carga modular que permita la carga simultánea de múltiples drones?
- ¿Qué tecnologías de localización y visión pueden ser implementadas para asegurar un aterrizaje preciso?

\section{Objetivos}

\subsection{Objetivo General}

Diseñar, manufacturar e instrumentar una base de carga inalámbrica para cuadricópteros de arquitectura abierta, y desarrollar un sistema de aterrizaje mediante el uso de sensores de localización y un sistema de visión, para asegurar una integración eficaz y segura entre el cuadricóptero y su sistema de carga.

\subsection{Objetivos Específicos}

\begin{itemize}
    \item Realizar un diseño CAD para la base de carga inalámbrica, asegurando que sea compatible con cuadricópteros de arquitectura abierta y que cumpla con los requisitos de eficiencia y seguridad para la carga de baterías.
    \item Manufacturar la base de carga para el enjambre de drones.
    \item Adaptar el diseño actual del dron de arquitectura abierta a la base de carga.
    \item Manufacturar un cuadricóptero e instrumentarlo con una cámara de visión y sensores de localización.
    \item Programar un sistema de aterrizaje que permita una buena estabilización y posicionamiento del cuadricóptero durante el aterrizaje, asegurando una interacción segura con la base de carga.
\end{itemize}

\section{Justificación}

\subsection{Relevancia del Proyecto}

Este proyecto es importante porque busca resolver un problema crucial en la operación continua y autónoma de enjambres de drones. Al mejorar la eficiencia de la carga y la precisión del aterrizaje, se contribuye al desarrollo de aplicaciones autónomas más seguras y eficientes en diversas industrias.

\subsection{Impacto Potencial}

Los resultados de este proyecto podrían tener un impacto significativo en la industria de la logística, la vigilancia y la agricultura de precisión, permitiendo la operación autónoma y segura de drones sin necesidad de intervención humana constante, lo cual aumentaría la eficiencia operativa y reduciría costos.

\section{Alcance y Limitaciones}

\subsection{Alcance del Trabajo}

Este trabajo se centrará en el diseño, manufactura e instrumentación de una base de carga inalámbrica para cuadricópteros, y en el desarrollo de un sistema de aterrizaje preciso utilizando sensores y visión por computadora. Las pruebas se llevarán a cabo en un entorno controlado.

\subsection{Limitaciones del Estudio}

El proyecto no considerará la escalabilidad a entornos exteriores con condiciones climáticas adversas ni la integración con otros tipos de drones que no sean de arquitectura abierta. Además, se limitará a la evaluación de prototipos en un entorno de prueba específico.

\section{Metodología General}

\subsection{Enfoque Metodológico}

Se utilizará un enfoque de diseño iterativo para construir el prototipo de la estación de carga. Las fases incluirán diseño CAD, manufactura, instrumentación de sensores, y pruebas experimentales para validar el aterrizaje y la carga de los drones.

\subsection{Herramientas y Tecnologías}

Las principales herramientas y tecnologías incluyen software CAD para el diseño, impresión 3D para la manufactura de componentes, sensores de visión por computadora para la localización, y algoritmos de control para la estabilización durante el aterrizaje.

\section{Estructura de la Tesis}

El capítulo 2 presentará una revisión detallada del estado del arte en sistemas de carga para drones. El capítulo 3 describirá la metodología empleada para el desarrollo del prototipo. El capítulo 4 discutirá los resultados obtenidos durante las pruebas, y el capítulo 5 incluirá las conclusiones y recomendaciones para futuros trabajos.

