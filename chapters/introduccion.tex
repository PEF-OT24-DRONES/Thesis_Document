\section{Problem Statement}

\subsection{Clear Problem Definition}

One of the main challenges in achieving continuous and autonomous operation of drone swarms is the lack of modular and autonomous charging stations capable of simultaneously recharging multiple drones. Current market solutions face significant limitations, such as the inability to efficiently manage the simultaneous charging of several drones and ensure precise landing in a shared space without human intervention. These limitations not only reduce drone autonomy but also impact operational efficiency, requiring frequent manual interventions to maneuver drones and taking up additional space due to the lack of modular solutions in existing market offerings.

\subsection{Research Questions}

\begin{itemize}
    \item How can a modular charging station be developed to enable simultaneous charging of multiple drones?
    \item What localization and vision technologies can be implemented to ensure precise landing?
    \item What resources are required to implement a precise landing and charging station for a drone swarm?
    \item Which charging method is the most efficient and effective for drone recharging in terms of performance and reliability?
\end{itemize}

\section{Objectives}

\subsection{General Objective}

To design and manufacture a charging station for open-architecture quadcopters, develop a communication system between the drone and the charging station, and instrument the drone with localization sensors and a vision system to ensure effective and safe integration between the quadcopter and its charging system.

\subsection{Specific Objectives}

\begin{itemize}
    \item Design a CAD model for the charging station, ensuring compatibility with open-architecture quadcopters.
    \item Manufacture a charging station module for the drone swarm.
    \item Adapt the current design of the open-architecture drone to the charging station.
    \item Equip a quadcopter with an onboard computer, flight controller, vision camera, and localization sensors.
    \item Develop a communication system between the charging station and the quadcopter for the transfer of relevant interaction data.
    \item Program a system to control the drone from an onboard computer.
    \item Develop a computer vision system capable of detecting in real-time the pose of an embedded ArUco marker on the charging station.
\end{itemize}

\section{Justification}

\subsection{Project Relevance}

The continuous and autonomous operation of drone swarms faces various challenges, including the lack of efficient charging solutions that allow the simultaneous recharging of multiple drones autonomously and without human intervention. While this project does not aim to fully solve this challenge, it lays the foundational groundwork for developing a viable solution. By designing a modular and autonomous charging station with a precise landing system, this project provides a starting point for future research and improvements.

The project is particularly relevant to industries such as logistics, surveillance, and precision agriculture, where autonomous drone operations can generate significant benefits. By presenting a modular solution capable of recharging multiple drones within the same physical space, this work sets the stage for future large-scale implementations, potentially reducing operational costs and improving efficiency \cite{cite1}. Although the problem is not entirely resolved in this work, the advancements presented here contribute to the development of practical solutions in the future.

\subsection{Potential Impact}

Although the system developed in this project is not a final solution, its implementation has the potential to influence future research and developments in the field of drone autonomy. The precise landing and simultaneous charging technologies explored in this work could be fundamental in designing more advanced and scalable solutions to meet the operational needs of industries increasingly relying on drones.

The potential impact of this project lies in its ability to inspire research that improves drone swarm efficiency in applications such as parcel delivery, large-area surveillance, and resource optimization in agriculture. While current solutions face limitations, this work provides a solid platform on which significant improvements can be implemented in future stages \cite{cite2}.

\section{Scope and Limitations}

\subsection{Project Scope}

This project establishes the foundations for a modular and autonomous drone recharging solution. It focuses on the design, manufacture, and instrumentation of a charging station capable of managing multiple drones simultaneously, utilizing computer vision and localization sensors. The work will be conducted in a controlled environment and validated under simulated conditions to ensure the solution is technically feasible.

While the project does not address autonomous large-scale charging, it provides a preliminary step towards a more advanced solution. The system will include the design of a modular charging station capable of recharging multiple drones in the same space, with a functional prototype of one charging module being physically developed for a single drone. Testing will be carried out on an open-architecture quadcopter, incorporating design modifications to facilitate charging and accommodate the vision and localization sensors \cite{cite3}.

\subsection{Study Limitations}

This study is limited to the technical validation of the system in a controlled environment with a specific type of drone (open-architecture quadcopter). It does not address scalability to other types of drones or implementation under adverse weather conditions. Additionally, the project does not include prolonged testing or commercial application implementation.

The vision and localization technologies implemented will also be limited to the test conditions, and their accuracy in more complex scenarios will not be evaluated in this work. The objective is to demonstrate the technical feasibility of the key components rather than delivering a fully functional final solution \cite{cite4}.

\section{Thesis Structure}

Chapter 2 will present a detailed review of the state of the art in drone charging systems, covering mechanics, electronics, and programming. Chapter 3 will describe the methodology employed in developing the prototype. Chapter 4 will discuss the results obtained during testing, and Chapter 5 will include conclusions and recommendations for future work.
