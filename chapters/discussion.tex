
In this section, the similarities and differences with other drone charging stations available on the market are discussed, highlighting the improvements implemented in our design, such as modularity and the automated opening system. Additionally, disadvantages compared to existing systems are analyzed, identifying potential areas for improvement.

Compared to other charging stations, our station was designed with a modular approach that allows for a stackable configuration, maximizing space utilization and facilitating system scalability. This contrasts with conventional stations, such as the DJI Dock 2, which, despite having advanced automation features and a compact design, does not offer the modular flexibility of our charging station. This limitation could restrict its expansion capacity in environments where space is a significant constraint \cite{bene2023}. Moreover, the DJI Dock 2 includes features such as a real-time kinematic positioning module (RTK) and wind sensors, which significantly enhance landing precision and resistance to adverse conditions—features that have not yet been implemented in our station \cite{nieuwoudt2023}.

A significant improvement in our design is the incorporation of an ArUco marker on the charging platform for visual detection and precise drone alignment during landing, similar to the technology used in the DJI Dock 2. This feature enables effective operation under different environmental conditions and overcomes the limitations of GPS positioning in enclosed environments where satellite signals are unreliable. Previous studies have demonstrated the effectiveness of vision-based systems in improving landing precision \cite{jiang2019}, highlighting the advantages of combining visual detection technologies with GPS positioning. However, our system still heavily depends on lighting conditions for the effective operation of the ArUco marker, representing a disadvantage in low-light environments. In contrast, the DJI Dock 2 system can operate in varied environmental conditions due to its night vision sensor \cite{nieuwoudt2023}.

Another relevant aspect is the charging method. Our station employs a charging system with magnetic coupling between the drone's legs and the mechanical guides of the charging station, facilitating landing and ensuring an efficient connection. Advanced solutions, such as wireless charging stations developed by MIT \cite{jiang2019} or automated battery replacement systems described in the literature \cite{grlj_docking_stations, desilva2022}, offer advantages in terms of recharge time and operational efficiency. Charging stations with fast-charging capabilities and automatic battery replacement systems, such as the inverted docking station proposed by De Silva et al. \cite{desilva2023}, are better suited for prolonged missions where flight interruptions must be minimized. Our modular charging station does not currently include a battery replacement system, which could be an interesting improvement to increase drone autonomy and reduce downtime.

Regarding comparisons with other types of charging stations, such as docking stations for security drones, our station presents certain advantages by being designed with a stackable structure that maximizes space utilization and increases charging capacity without expanding the installation area \cite{grlj_docking_stations}. Additionally, we integrated automated communication that enables the transmission and real-time processing of information captured by the drone, simplifying the management of security and surveillance missions \cite{nieuwoudt2023}.

Finally, optimizing the communication system between the drone and the charging station could increase the efficiency of the landing and charging process, especially when operating with multiple stations in a swarm environment, as suggested in the context of swarm drone control \cite{marek2023}. Furthermore, a possible improvement would be the incorporation of additional sensors into the charging stations, such as anemometers and temperature sensors, to provide real-time data to the drone during landing in uncontrolled spaces \cite{bene2023, nieuwoudt2023}. This would allow the drone to adapt more precisely to environmental conditions, thereby increasing landing safety and effectiveness. Similarly, implementing a more drone-friendly charging solution, such as a wireless charging system similar to the DJI Dock 2 \cite{nieuwoudt2023}, could improve efficiency and reduce equipment wear.


