
En esta sección, se concluye todo lo que se realizó durante el desarrollo de la tesis, desde el diseño y la fabricación de la estación de carga modular hasta la instrumentación del dron con sensores de localización y un sistema de visión por computadora. Se logró cumplir con los objetivos planteados, destacando la modularidad y eficiencia del sistema de carga propuesto, así como la integración exitosa del dron con la estación de carga.

A pesar de los logros alcanzados, se identificaron áreas de mejora que podrían explorarse en trabajos futuros. Entre estas mejoras se encuentra la implementación de un sistema de carga rápida, que permitiría reducir significativamente el tiempo de carga del dron. Asimismo, sería beneficioso desarrollar un sistema de iluminación autónoma para garantizar un aterrizaje preciso en condiciones de poca luz, mejorando así la autonomía operativa del dron en diferentes escenarios.

Finalmente, la optimización del sistema de comunicación entre el dron y la estación de carga podría incrementar la eficiencia del proceso de aterrizaje y carga, especialmente cuando se opera con múltiples estaciones en un entorno de enjambre.