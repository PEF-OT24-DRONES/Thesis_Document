This thesis has successfully demonstrated the design, fabrication, and testing of a modular charging station integrated with an instrumented drone equipped with localization sensors and a computer vision system. The project addressed the critical challenges of scalability, modularity, and compatibility in autonomous drone charging systems, achieving significant milestones while identifying areas for future enhancement.

The modular charging station was designed with a robust and reliable V-belt and pulley mechanism for drawer movement, capable of supporting repetitive operations without failure. Its stackable design allows scalability, enabling deployment in swarm scenarios where multiple drones can utilize the system simultaneously. The instrumented drone, built on an open-source F450 frame, was enhanced with custom 3D-printed components that improved its structural adaptability and ensured seamless integration with the charging station. The combination of the modular station and the customized drone resulted in a system that is versatile, efficient, and adaptable for diverse applications.

In terms of operational performance, the drone demonstrated stable flight and precise outdoor positioning using GPS, validating the instrumentation's reliability. The vision system successfully detected and calculated the position and orientation of ArUco markers, even under challenging conditions such as varying lighting and partial occlusion. This capability proved essential for aligning the drone with the charging station and showed potential for future autonomous docking implementations. However, indoor flight tests could not be conducted due to the absence of indoor positioning sensors, highlighting a limitation of the current design.

The charging system performed well in stationary conditions, slowing battery discharge significantly. However, the charger’s limited capacity prevented full recharging of the drone’s battery during operations, emphasizing the need for a higher-capacity charging solution. Additionally, the communication system, built on ROS 2 and leveraging a local WiFi network, ensured stable and efficient data exchange between the drone, the station, and the ground computer. This system successfully coordinated real-time operations, establishing a strong foundation for multi-device interactions.

Despite these accomplishments, the project identified several opportunities for improvement. Integrating a fast-charging system would reduce downtime, increasing the drone’s operational efficiency. Autonomous lighting and guidance systems could enhance precision in docking, especially under low-light conditions, while optimizing the communication system for swarm operations could allow simultaneous use of multiple drones and stations with minimal interference.

Overall, this thesis lays a solid foundation for modular drone charging systems, addressing critical design challenges and proving the feasibility of scalable and adaptable solutions. The results demonstrate the system’s potential for various industrial applications, including surveillance, agriculture, and logistics. By incorporating the proposed enhancements, this system could evolve into a fully autonomous platform capable of supporting large-scale drone swarm operations, significantly advancing the field of autonomous aerial systems.